\documentclass[a4paper,oneside,11pt]{article}
\usepackage[utf8]{inputenc}

\usepackage[ngerman]{babel} % For correct hyphenation

\usepackage{mathtools}
\usepackage{amssymb,amsmath,amsthm}
\usepackage[mathscr]{eucal}

\usepackage[textwidth=17cm,top=1.5cm,bottom=1.5cm,nohead]{geometry}

\setlength{\parindent}{0mm} % no paragraph indentation


%%%%%%%%%%%%%%%%%%%%%%%%%%%%%%%%%%%%%%%%%%%%%%%%%%%%%%%

% Abbreviations

%%%%%%%%%%%%%%%%%%%%%%%%%%%%%%%%%%%%%%%%%%%%%%%%%%%%%%%
% single letters in different fonts 

%%%%%%%%%%%% mathematical bold  %%%%%%%%%%%%%%%%%%%%

\newcommand{\bA}{\mathbb{A}}
\newcommand{\bB}{\mathbb{B}}
\newcommand{\bC}{\mathbb{C}}
\newcommand{\bD}{\mathbb{D}}
\newcommand{\bE}{\mathbb{E}}
\newcommand{\bF}{\mathbb{F}}
\newcommand{\bG}{\mathbb{G}}
\newcommand{\bH}{\mathbb{H}}
\newcommand{\bI}{\mathbb{I}}
\newcommand{\bJ}{\mathbb{J}}
\newcommand{\bK}{\mathbb{K}}
\newcommand{\bL}{\mathbb{L}}
\newcommand{\bM}{\mathbb{M}}
\newcommand{\bN}{\mathbb{N}}
\newcommand{\bO}{\mathbb{O}}
\newcommand{\bP}{\mathbb{P}}
\newcommand{\bQ}{\mathbb{Q}}
\newcommand{\bR}{\mathbb{R}}
\newcommand{\bS}{\mathbb{S}}
\newcommand{\bT}{\mathbb{T}}
\newcommand{\bU}{\mathbb{U}}
\newcommand{\bV}{\mathbb{V}}
\newcommand{\bW}{\mathbb{W}}
\newcommand{\bX}{\mathbb{X}}
\newcommand{\bY}{\mathbb{Y}}
\newcommand{\bZ}{\mathbb{Z}}


%%%%%%%%% calligraphic %%%%%%%%%%%%%%%%%%%%%%%
\newcommand{\mc}[1]{\mathcal{#1}}

\newcommand{\cA}{\mathcal{A}}
\newcommand{\cB}{\mathcal{B}}
\newcommand{\cC}{\mathcal{C}}
\newcommand{\cD}{\mathcal{D}}
\newcommand{\cE}{\mathcal{E}}
\newcommand{\cF}{\mathcal{F}}
\newcommand{\cG}{\mathcal{G}}
\newcommand{\cH}{\mathcal{H}}
\newcommand{\cI}{\mathcal{I}}
\newcommand{\cJ}{\mathcal{J}}
\newcommand{\cK}{\mathcal{K}}
\newcommand{\cL}{\mathcal{L}}
\newcommand{\cM}{\mathcal{M}}
\newcommand{\cN}{\mathcal{N}}
\newcommand{\cO}{\mathcal{O}}
\newcommand{\cP}{\mathcal{P}}
\newcommand{\cQ}{\mathcal{Q}}
\newcommand{\cR}{\mathcal{R}}
\newcommand{\cS}{\mathcal{S}}
\newcommand{\cT}{\mathcal{T}}
\newcommand{\cU}{\mathcal{U}}
\newcommand{\cV}{\mathcal{V}}
\newcommand{\cW}{\mathcal{W}}
\newcommand{\cX}{\mathcal{X}}
\newcommand{\cY}{\mathcal{Y}}
\newcommand{\cZ}{\mathcal{Z}}


%%%%%%%%%%%%% mathematical fraktur  %%%%%%%%%%%%%%%%%%%%%
\newcommand{\mf}[1]{\mathfrak{#1}}

\newcommand{\fA}{\mathfrak{A}}
\newcommand{\fB}{\mathfrak{B}}
\newcommand{\fC}{\mathfrak{C}}
\newcommand{\fD}{\mathfrak{D}}
\newcommand{\fE}{\mathfrak{E}}
\newcommand{\fF}{\mathfrak{F}}
\newcommand{\fG}{\mathfrak{G}}
\newcommand{\fH}{\mathfrak{H}}
\newcommand{\fI}{\mathfrak{I}}
\newcommand{\fJ}{\mathfrak{J}}
\newcommand{\fK}{\mathfrak{K}}
\newcommand{\fL}{\mathfrak{L}}
\newcommand{\fM}{\mathfrak{M}}
\newcommand{\fN}{\mathfrak{N}}
\newcommand{\fO}{\mathfrak{O}}
\newcommand{\fP}{\mathfrak{P}}
\newcommand{\fQ}{\mathfrak{Q}}
\newcommand{\fR}{\mathfrak{R}}
\newcommand{\fS}{\mathfrak{S}}
\newcommand{\fT}{\mathfrak{T}}
\newcommand{\fU}{\mathfrak{U}}
\newcommand{\fV}{\mathfrak{V}}
\newcommand{\fW}{\mathfrak{W}}
\newcommand{\fX}{\mathfrak{X}}
\newcommand{\fY}{\mathfrak{Y}}
\newcommand{\fZ}{\mathfrak{Z}}


%%%%%%%%%%%%% mathematical script (euler)  %%%%%%%%%%%%%%%%%%%%%
\newcommand{\ms}[1]{\mathscr{#1}}

\newcommand{\sA}{\mathscr{A}}
\newcommand{\sB}{\mathscr{B}}
\newcommand{\sC}{\mathscr{C}}
\newcommand{\sD}{\mathscr{D}}
\newcommand{\sE}{\mathscr{E}}
\newcommand{\sF}{\mathscr{F}}
\newcommand{\sG}{\mathscr{G}}
\newcommand{\sH}{\mathscr{H}}
\newcommand{\sI}{\mathscr{I}}
\newcommand{\sJ}{\mathscr{J}}
\newcommand{\sK}{\mathscr{K}}
\newcommand{\sL}{\mathscr{L}}
\newcommand{\sM}{\mathscr{M}}
\newcommand{\sN}{\mathscr{N}}
\newcommand{\sO}{\mathscr{O}}
\newcommand{\sP}{\mathscr{P}}
\newcommand{\sQ}{\mathscr{Q}}
\newcommand{\sR}{\mathscr{R}}
\newcommand{\sS}{\mathscr{S}}
\newcommand{\sT}{\mathscr{T}}
\newcommand{\sU}{\mathscr{U}}
\newcommand{\sV}{\mathscr{V}}
\newcommand{\sW}{\mathscr{W}}
\newcommand{\sX}{\mathscr{X}}
\newcommand{\sY}{\mathscr{Y}}
\newcommand{\sZ}{\mathscr{Z}}


%%%%%%%%%%    Math operators    %%%%%%%%%%%%%%%%%%%%%%%%%%%

\renewcommand{\Re}{\mathop{\textnormal{Re}}}  % real part
\renewcommand{\Im}{\mathop{\textnormal{Im}}}  % imaginary part


%%%%%%%%%%%  FURTHER COMMANDS  %%%%%%%%%%%%%%%

\newcommand{\Id}{\mathrm{Id}}


%%%%%%%%%%%  STUDENT COMMANDS  %%%%%%%%%%%%%%%
%% Hier können Sie Ihre eigene LaTeX kommandos hinzufügen. %%
\newtheorem*{theorem*}{Theorem}
%% this allows for theorems which are not automatically numbered
\newtheorem{definition}{Definition}
\newtheorem{theorem}{Theorem}
\newtheorem{lemma}{Lemma}
\newtheorem{example}{Example}

%%%%%%%%%%%%%%%%%%%%%%%%%%%%%%%%%%%%%%%%%%%


\begin{document}
{\Large\bf Nr 8.5}\\\\
\hrule
%%%%%%%%%%%%%%%%%%%%%%%%%%%%%%%%%%%%%%%
\bigskip
Beweis durch Widerspruch. Sei $T$ ist entweder injektiv, aber nicht surjektiv, oder surjektiv, aber nicht injektiv.\\
Sei $\dim V=n$.\bigskip

$V$ hat n Basisvektoren $f_{1},\dots ,f_{n}$.\\
Demnach existiert ein Isomorphismus $A:V\to\mathbb{K}^{n}$ zwischen $V$ und $\mathbb{K}{n}$, wobei $A(f_{i})=e_{i}, i\leq n$
\bigskip

Sei $T$ ist injektiv, aber nicht surjektiv.\\
Dann $A\circ T\circ A^{-1}:\mathbb{K}^{n}\to \mathbb{K}^{n}$ ist injektiv, aber nicht surjektiv.\bigskip

Beweis:\\
Injektivität:
\begin{align*}
	 & \textnormal{Sei } x,y\in \mathbb{K}^{n}, x\neq y.                 \\
	 & A^{-1}(x)\neq A^{-1}(y) \textnormal{( Da $A^{-1}$ injektiv)}      \\
	 & T(A^{-1}(x))\neq  T(A^{-1}(y)) \textnormal{ (da T injektiv)}      \\
	 & A(T(A^{-1}(x)))\neq A(T(A^{-1}(y))) \textnormal{ (da A injektiv)}
\end{align*}
nicht surjektiv:
\begin{align*}
	 & \exists a\in V: \not \exists b\in V: T(b)=a \textnormal{ (Da T nicht surjektiv)}                                                                                \\
	 & \exists x\in \mathbb{K}^{n}:a= A^{-1}(x) \textnormal{ (Da $A^{-1}$ surjektiv)}                                                                                  \\
	 & \exists x\in \mathbb{K}^{n} : \not \exists b\in V: A(T(b))=A(A^{-1}(x))=x \textnormal{ (Da A injektiv)}                                                         \\
	 & \exists x\in \mathbb{K}^{n}: \not \exists y\in \mathbb{K}^{n}:A(T(A^{-1}(y)))=x \textnormal{\textnormal{ (Da $A^{-1}(y)=b\in V, \forall y\in \mathbb{K}^{n}$)}} \\
\end{align*}




Sei $T$ ist surjektiv, aber nicht injektiv.\medskip

Dann $A\circ T\circ A^{-1}$ ist surjektiv, aber nicht injektiv.\bigskip

Beweis:\\
nicht Injektiv:
\begin{align*}
	 & \textnormal{Da T nicht injektiv gibt es ein $a,b\in V: a\neq  b, T(a)=T(b)$}          \\
	 & \exists x,y, x\neq y: T(A^{-1}(x))=T(A^{-1}(y)),  \textnormal{ da $A^{-1}$ injektiv.} \\
	 & A(T(A^{-1}(x)))=A(T(A^{-1}(y))), x\neq y, \textnormal{da A injektiv}
\end{align*}
Surjektiv:\\
\begin{align*}
	 & a = T(b), \forall a\in V \textnormal{ (da T surjektiv)}                                   \\
	 & A^{-1}(x)=T(A^{-1}(y)), \forall x\in \mathbb{K}^{n} \textnormal{ (da $A^{-1}$ surjektiv)} \\
	 & x= A(A^{-1}(x))=A(T(A^{-1}(y))), \forall x\in \mathbb{K}^{n}
\end{align*}
Dies ist ein Widerspuch  zu dem Satz, dass jede lineare Abbildung $T:\mathbb{K}^{n}\to \mathbb{K}^{n}$ surektiv ist g.d.w. sie injektiv ist.\\
Somit muss der Satz auch für eine Beliebige lineare Abbildung $T:V\to V$ gelten.
\end{document}

