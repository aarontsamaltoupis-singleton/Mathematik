\documentclass{article}
\usepackage{graphicx} % Required for inserting images
\usepackage{amsmath}
\usepackage{amsthm}
\usepackage{amssymb}
\usepackage{stmaryrd}


\title{}
\author{
    Aaron Tsamaltoupis,
    Matr.Nr.: 3762396\\
}
\date{\today}

\newcommand{\ZZ}{\mathrm{Z\kern-.3em\raise-0.5ex\hbox{Z}}}

\begin{document}
\maketitle
\section*{3}
\subsection*{2)}
Es gilt:
$$
	e = \lim \limits_{k\to\infty}(1+\frac{1}{k})^k
$$
wobei die Folge $((1+\frac{1}{k})^k)_k$ monoton steigend ist.\\
Es gilt $e> (1+\frac{1}{n})^n, \forall n\in\mathbb{N} $

\begin{align}
	\left(\frac{n+1}{n}\right)^{n+1}-e = &  &
	\left(\frac{n+1}{n}\right)\left(\frac{n+1}{n}\right)^n-e                                                                        \\
	                                     &  & < \left(\frac{n+1}{n}\right)\left(\frac{n+1}{n}\right)^n-\left(\frac{n+1}{n}\right)^n \\
	                                     &  & =  \left(\frac{n+1}{n}\right)^n\left(1+\frac{1}{n} -1\right)                          \\
	                                     &  & = \left(\frac{n+1}{n}\right)^n\cdot \frac{1}{n}                                       \\
	                                     &  & = e\cdot \frac{1}{n}
\end{align}
Somit gilt:
\begin{align}
	\left(\left(\frac{n+1}{n}\right)^{n+1} - e\right)^2 \\
	<(e\cdot \frac{1}{n})^2                             \\
	\leq e^2\cdot \frac{1}{n^2}
\end{align}
$\sum_{=1}^{\infty} e^2 \cdot \frac{1}{n^2}$ omst somit eine Majorante der Reihe $\sum_n^\infty \left(\left(\frac{n+1}{n}\right)^{n+1}-e\right)^2
$.\\
Da $\sum^{\infty}_{n=1}\frac{1}{n^2}$ konvergiert, konvergiert nach den Rechenregeln für konvergente Folgen auch die Reihe $\sum_{n=1}^\infty e^2 \cdot  \frac{1}{n^2} = e^2 \cdot \sum_{n=1}^\infty \frac{1}{n^2}$.\\

Nach dem Majorantenkriterium konvergiert damit auch die Reihe $$
	\sum_n^\infty \left(\left(\frac{n+1}{n}\right)^{n+1}-e\right)^2
$$
\end{document}

