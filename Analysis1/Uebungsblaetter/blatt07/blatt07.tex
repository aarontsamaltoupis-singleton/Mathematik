\documentclass{article}
\usepackage{graphicx} % Required for inserting images
\usepackage{amsmath}
\usepackage{amsthm}
\usepackage{amssymb}
\usepackage{stmaryrd}


\title{}
\author{
    Aaron Tsamaltoupis,
    Matr.Nr.: 3762396\\
}
\date{\today}

\newcommand{\ZZ}{\mathrm{Z\kern-.3em\raise-0.5ex\hbox{Z}}}

\begin{document}
\maketitle
\section*{3.1}
\subsection*{2)}
Es gilt:
$$
	e = \lim \limits_{k\to\infty}(1+\frac{1}{k})^k
$$
wobei die Folge $((1+\frac{1}{k})^k)_k$ monoton steigend ist.\\
Es gilt $(1+\frac{1}{n})^{n+1}>e> (1+\frac{1}{n})^n, \forall n\in\mathbb{N} $

\begin{align}
	0<\left(\frac{n+1}{n}\right)^{n+1}-e = &
	\left(\frac{n+1}{n}\right)\left(\frac{n+1}{n}\right)^n-e                                                                       \\
	                                       & < \left(\frac{n+1}{n}\right)\left(\frac{n+1}{n}\right)^n-\left(\frac{n+1}{n}\right)^n \\
	%  &=  \left(\frac{n+1}{n}\right)^n\left(1+\frac{1}{n} -1\right)                          \\
	                                       & = \left(\frac{n+1}{n}\right)^n\cdot \frac{1}{n}                                       \\
	                                       & < e\cdot \frac{1}{n}
\end{align}
Somit gilt:
\begin{align}
	\left(\left(\frac{n+1}{n}\right)^{n+1} - e\right)^2
	%	<(e\cdot \frac{1}{n})^2                             \\
	\leq e^2\cdot \frac{1}{n^2}
\end{align}
$\sum\limits_{n=1}^{\infty} e^2 \cdot \frac{1}{n^2}$ ist somit eine Majorante der Reihe $\sum\limits_n^\infty \left(\left(\frac{n+1}{n}\right)^{n+1}-e\right)^2
$.\\
Da $\sum\limits^{\infty}_{n=1}\frac{1}{n^2}$ konvergiert, konvergiert nach den Rechenregeln für konvergente Folgen auch die Reihe $\sum\limits_{n=1}^\infty e^2 \cdot  \frac{1}{n^2} = e^2 \cdot \sum\limits_{n=1}^\infty \frac{1}{n^2}$.\\

Nach dem Majorantenkriterium konvergiert damit auch die Reihe $$
	\sum_n^\infty \left(\left(\frac{n+1}{n}\right)^{n+1}-e\right)^2
$$
\section*{3.4}
Es gilt:
$$\{1,2,\dots ,m-(d+1)\}\subseteq \{\tau (1), \dots ,\tau (m)\}\subseteq \{1,\dots ,m+d\}, \forall m\in\mathbb{N} $$
Da $(a_n)_n$ eine Nullfolge ist, kann ein N gewählt werden, sodass für alle $n>N$ gilt: $$a_n<\frac{\varepsilon}{4\cdot d} $$\\
Nach dem Cauchy-Kriterium kann  ein $M\in\mathbb{N} $ gewählt werden, sodass $$\forall m>n>M:\sum_{k=n}^m a_k<\frac{\varepsilon}{2} $$.\\
Sei $m>\max\{N,M\}+d$
\begin{align*}
	\left|\sum_{n=1}^m a_{\tau (n)}-\sum^m_{n=1}a_n\right| & = \left|\sum^{m-d-1}_{n=1}a_n\ +\sum^{m}_{n=m-d}a_{ \tau (n)}-\sum^m_{n=1}a_n\right| \\
	                                                       & \leq \left|\sum^{m}_{n=m-d}a_n\ \right|+\sum^{m}_{n=m-d}|a_{ \tau (n)}|              \\
	                                                       & \leq \left|\sum^{m}_{n=m-d}a_n\right| + \sum^{m+d}_{n=m-d} |a_n|                     \\
	                                                       & \leq \frac{\varepsilon }{2}+ 2d\cdot \frac{ \varepsilon}{4d}\leq \varepsilon
\end{align*}
Da die Folgen einen beliebig kleinen Abstand haben, konvergieren sie zu dem selben Grenzwert.

\newpage
\section*{3.5}
Sei $C_n=\sum_{k=0}^nc_k$,
$B_n=\sum_{k=0}^nb_k$,
$A_n=\sum_{k=0}^na_k$\\

\subsection*{Lemma 1}
$$\sum_{n=0}^m C_n = \sum_{n=0}^m A_{m-n}B_n$$\\

\subsubsection*{Beweis Lemma 1}
\begin{align*}
	C_m = & \sum^m_{n=0}\sum^n_{k=0}a_k\cdot b_{n-k} \\
	%& = \sum_{n=0}^m b_n \sum_{k=n}^m a_{n-k}  \\
	%& = \sum_{n=0}^m b_n \sum_{k=0}^{m-n} a_n  \\
	      & = \sum_{n=0}^m b_n A_{m-n}
\end{align*}\\
Beweis durch Induktion, dass
$$\sum_{n=0}^m \sum^n_{k=0}b_k A_{n-k}=
	\sum_{n=0}^m A_{m-n}B_n$$
\textbf{Induktionsanfang}\\
Sei $m=0$.
$$\sum^0_{n=0}\sum^n_{k=0}b_nA_{0-n} = b_0A_0 = B_0A_0 = \sum_{n=0}^0 A_{0-n}B_n$$
\textbf{Induktionsschritt}\\
Induktionshypothese: Sei
$$\sum_{n=0}^m \sum^n_{k=0}b_k A_{n-k}=
	\sum_{n=0}^m A_{m-n}B_n$$


\begin{align*}
	\sum^{m+1}_{n=0}\sum^n_{k=0}b_k A_{n-k} & =\sum^{m}_{n=0}\sum^n_{k=0}b_k A_{n-k}+\sum^{m+1}_{n=0}b_n A_{m+1-n}                \\
	                                        & \overset{IH}{=}\sum_{n=0}^m A_{m-n}B_n+ b_0 A_{m+1} + \sum^{m+1}_{n=1}b_n A_{m+1-n} \\
	                                        & = \sum^m _{n=0} A_{m-n} B_n + \sum^m_{n=0}b_{n+1}A_{m-n} +b_0 A_{m+1}               \\
	                                        & = \sum^m _{n=0} A_{m-n} B_n + b_{n+1}A_{m-n} +b_0 A_{m+1}                           \\
	                                        & =\sum^m_{n=0} A_{m-n}(B_{n+1})+(-A_0 B_{m+1} + A_0 B_{m+1}) + b_0 A_{m+1}           \\
	                                        & = \sum^{m-1}_{n=0} A_{m-n} B_{n+1} + A_0 B_{m+1}+b_0A_{m+1}                         \\
	                                        & = \sum^{m}_{n=1} A_{m+1-n} B_{n} + A_0 B_{m+1}+b_0A_{m+1}                           \\
	                                        & = \sum^{m+1}_{n=0}A_{m+1-n}B_n
\end{align*}

Da $C_m = \sum_{n=0}^m b_nA_{m-n}$,  gilt
$$
	\sum_{n=0}^m C_m = \sum_{n=0}^m b_n A_{m-n}\qed
$$
\subsection*{Lemma 2}
Aus $d_n\to e, e_n \to e$ folgt:
$$
	\frac{1}{n+1}\sum_{k=0}^n d_{n-k}e_k\to de
$$
- leider kein Beweis
\subsubsection*{Hilfssatz1}\\
Sei ein $\varepsilon >0$
Nach Lemma 2 kann ein $M_0$ gefunden werden, sodass $\forall m > M_0:$
\begin{align}
	|\frac{1}{m+1}\sum_{n=0}^m A_{m-n}B_n-AB|<\frac{\varepsilon }{4}                \\
	(m+1)\cdot |\frac{1}{m+1}\sum_{n=0}^m A_{m-n}B_n-AB|<(m+1)\frac{\varepsilon}{4} \\
	|\sum_{n=0}^m A_{m-n}B_n-(m+1)AB|<(m+1)\frac{\varepsilon}{4}                    \\
\end{align}
Nach Lemma 1 gilt:
$$
	|\sum^m_{n=0}C_n-(m+1)AB|<(m+1)\varepsilon                         \\
$$
Speziell gilt außerdem:
$$
	|\sum^{2m}_{n=0}C_n-(2m+1)AB|<(2m+1)\varepsilon                         \\
$$
\\
\\
\textbf{Hilfssatz2}\\
$C_n$ konvergiert gegen einen Grenzwert $C$.\\
Es kann also für jedes $\varepsilon >0$ ein $M_1$ gefunden werden, sodass für alle $m_1>M_1$ gilt $|C-C_{m_1}|<\varepsilon$

Also gilt für alle $m_2>m_1>M_1$:
\begin{align*}
	|(m_2-m_1)C_{m_2}-\sum^{m_2}_{n=m_1}C_n|<\varepsilon
\end{align*}

\textbf{Hilfssatz3}\\
Da die Folge $\frac{m+1}{m}$ gegen 1 konvergiert, kann ein $M_2$ gewählt werden, sodass\\ $\forall m>M_2: |1-\frac{m+1}{m}|< \frac{\varepsilon}{4\cdot AB}$
\begin{align*}
	|AB-\frac{m+1}{m}AB|=|AB(1-\frac{m+1}{m})|\leq |AB(\frac{\varepsilon }{4AB})|\leq \frac{ \varepsilon}{4}
\end{align*}\\
\textbf{Hilfssatz4}\\
Wenn $|F+G|<k $ und $|F'-F|<k' $, dann $|F'+G|<k+k'$
$$k+k'>|F+G|+|F'-F|\geq |F+G+F'-F|=|F'+G|$$
\\\\
Sei $M= \max \{M_0,M_1,M_2,3\}$.
Für alle $m>M$ gilt dann:
\begin{align*}
	\text{Nach Hilfssatz1: }                                                                               \\
	 & |\sum^{2m}_{n=0}C_n-(2m+1)AB|                                                                       \\
	 & = |\sum^{m-1}_{n=0}C_n+\sum_{n=m}^{2m}C_n-(2m+1)AB|<(2m+1)\frac{\varepsilon }{4}                    \\
	 & \textnormal{Nach Hilfssatz 1, 2 und 4:}                                                             \\
	 & |m\cdot AB + mC_m - (2m+1) AB |<(2m+3) \frac{\varepsilon }{4}                                       \\
	 & \implies |mC_m - (m+1) AB |<(2m+3) \frac{\varepsilon }{4}                                           \\
	 & \implies |C_m -\frac{m+1}{m}AB| <(2+\frac{3}{m})\frac{\varepsilon }{4}                              \\
	 & \textnormal{Nach Hilfssatz 3 und 4:}                                                                \\
	 & |C_m - AB|< (3+\frac{3}{m})\frac{\varepsilon }{4}\leq 4\cdot \frac{\varepsilon }{4}\leq \varepsilon
\end{align*}
Somit konvergiert $C_m$ gegen $AB$, was zu zeigen war.
\end{document}

