\documentclass{article}
\usepackage{graphicx} % Required for inserting images
\usepackage{amsmath}
\usepackage{amsthm}
\usepackage{amssymb}
\usepackage{stmaryrd}


\title{}
\author{
    Aaron Tsamaltoupis,
    Matr.Nr.: 3762396\\
}
\date{\today}

\newcommand{\ZZ}{\mathrm{Z\kern-.3em\raise-0.5ex\hbox{Z}}}

\begin{document}
\maketitle
\section*{3.1}
\subsection*{2)}
Es gilt:
$$
	e = \lim \limits_{k\to\infty}(1+\frac{1}{k})^k
$$
wobei die Folge $((1+\frac{1}{k})^k)_k$ monoton steigend ist.\\
Es gilt $(1+\frac{1}{n})^{n+1}>e> (1+\frac{1}{n})^n, \forall n\in\mathbb{N} $

\begin{align}
	0<\left(\frac{n+1}{n}\right)^{n+1}-e = &
	\left(\frac{n+1}{n}\right)\left(\frac{n+1}{n}\right)^n-e                                                                       \\
	                                       & < \left(\frac{n+1}{n}\right)\left(\frac{n+1}{n}\right)^n-\left(\frac{n+1}{n}\right)^n \\
	%  &=  \left(\frac{n+1}{n}\right)^n\left(1+\frac{1}{n} -1\right)                          \\
	                                       & = \left(\frac{n+1}{n}\right)^n\cdot \frac{1}{n}                                       \\
	                                       & < e\cdot \frac{1}{n}
\end{align}
Somit gilt:
\begin{align}
	\left(\left(\frac{n+1}{n}\right)^{n+1} - e\right)^2
	%	<(e\cdot \frac{1}{n})^2                             \\
	\leq e^2\cdot \frac{1}{n^2}
\end{align}
$\sum\limits_{n=1}^{\infty} e^2 \cdot \frac{1}{n^2}$ ist somit eine Majorante der Reihe $\sum\limits_n^\infty \left(\left(\frac{n+1}{n}\right)^{n+1}-e\right)^2
$.\\
Da $\sum\limits^{\infty}_{n=1}\frac{1}{n^2}$ konvergiert, konvergiert nach den Rechenregeln für konvergente Folgen auch die Reihe $\sum\limits_{n=1}^\infty e^2 \cdot  \frac{1}{n^2} = e^2 \cdot \sum\limits_{n=1}^\infty \frac{1}{n^2}$.\\

Nach dem Majorantenkriterium konvergiert damit auch die Reihe $$
	\sum_n^\infty \left(\left(\frac{n+1}{n}\right)^{n+1}-e\right)^2
$$
\section*{3.4}
Es gilt:
$$\{1,2,\dots ,m-(d+1)\}\subseteq \{\tau (1), \dots ,\tau (m)\}\subseteq \{1,\dots ,m+d\}, \forall m\in\mathbb{N} $$
Da $(a_n)_n$ eine Nullfolge ist, kann ein N gewählt werden, sodass für alle $n>N$ gilt: $$a_n<\frac{\varepsilon}{4\cdot d} $$\\
Nach dem Cauchy-Kriterium kann  ein $M\in\mathbb{N} $ gewählt werden, sodass $$\forall m>n>M:\sum_{k=n}^m a_k<\frac{\varepsilon}{2} $$.\\
Sei $m>\max\{N,M\}+d$
\begin{align*}
	\left|\sum_{n=1}^m a_{\tau (n)}-\sum^m_{n=1}a_n\right| & = \left|\sum^{m-d-1}_{n=1}a_n\ +\sum^{m}_{n=m-d}a_{ \tau (n)}-\sum^m_{n=1}a_n\right| \\
	                                                       & \leq \left|\sum^{m}_{n=m-d}a_n\ \right|+\sum^{m}_{n=m-d}|a_{ \tau (n)}|              \\
	                                                       & \leq \left|\sum^{m}_{n=m-d}a_n\right| + \sum^{m+d}_{n=m-d} |a_n|                     \\
	                                                       & \leq \frac{\varepsilon }{2}+ 2d\cdot \frac{ \varepsilon}{4d}\leq \varepsilon
\end{align*}
Da die Folgen einen beliebig kleinen Abstand haben, konvergieren sie zu dem selben Grenzwert.
\end{document}

