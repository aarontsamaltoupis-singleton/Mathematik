
\documentclass[a4paper,oneside,11pt]{article}
\usepackage[utf8]{inputenc}

\usepackage[ngerman]{babel} % For correct hyphenation

\usepackage{mathtools}
\usepackage{amssymb,amsmath,amsthm}
\usepackage[mathscr]{eucal}

\usepackage[textwidth=17cm,top=1.5cm,bottom=1.5cm,nohead]{geometry}

\setlength{\parindent}{0mm} % no paragraph indentation


%%%%%%%%%%%%%%%%%%%%%%%%%%%%%%%%%%%%%%%%%%%%%%%%%%%%%%%

% Abbreviations

%%%%%%%%%%%%%%%%%%%%%%%%%%%%%%%%%%%%%%%%%%%%%%%%%%%%%%%
% single letters in different fonts 

%%%%%%%%%%%% mathematical bold  %%%%%%%%%%%%%%%%%%%%

\newcommand{\bA}{\mathbb{A}}
\newcommand{\bB}{\mathbb{B}}
\newcommand{\bC}{\mathbb{C}}
\newcommand{\bD}{\mathbb{D}}
\newcommand{\bE}{\mathbb{E}}
\newcommand{\bF}{\mathbb{F}}
\newcommand{\bG}{\mathbb{G}}
\newcommand{\bH}{\mathbb{H}}
\newcommand{\bI}{\mathbb{I}}
\newcommand{\bJ}{\mathbb{J}}
\newcommand{\bK}{\mathbb{K}}
\newcommand{\bL}{\mathbb{L}}
\newcommand{\bM}{\mathbb{M}}
\newcommand{\bN}{\mathbb{N}}
\newcommand{\bO}{\mathbb{O}}
\newcommand{\bP}{\mathbb{P}}
\newcommand{\bQ}{\mathbb{Q}}
\newcommand{\bR}{\mathbb{R}}
\newcommand{\bS}{\mathbb{S}}
\newcommand{\bT}{\mathbb{T}}
\newcommand{\bU}{\mathbb{U}}
\newcommand{\bV}{\mathbb{V}}
\newcommand{\bW}{\mathbb{W}}
\newcommand{\bX}{\mathbb{X}}
\newcommand{\bY}{\mathbb{Y}}
\newcommand{\bZ}{\mathbb{Z}}


%%%%%%%%% calligraphic %%%%%%%%%%%%%%%%%%%%%%%
\newcommand{\mc}[1]{\mathcal{#1}}

\newcommand{\cA}{\mathcal{A}}
\newcommand{\cB}{\mathcal{B}}
\newcommand{\cC}{\mathcal{C}}
\newcommand{\cD}{\mathcal{D}}
\newcommand{\cE}{\mathcal{E}}
\newcommand{\cF}{\mathcal{F}}
\newcommand{\cG}{\mathcal{G}}
\newcommand{\cH}{\mathcal{H}}
\newcommand{\cI}{\mathcal{I}}
\newcommand{\cJ}{\mathcal{J}}
\newcommand{\cK}{\mathcal{K}}
\newcommand{\cL}{\mathcal{L}}
\newcommand{\cM}{\mathcal{M}}
\newcommand{\cN}{\mathcal{N}}
\newcommand{\cO}{\mathcal{O}}
\newcommand{\cP}{\mathcal{P}}
\newcommand{\cQ}{\mathcal{Q}}
\newcommand{\cR}{\mathcal{R}}
\newcommand{\cS}{\mathcal{S}}
\newcommand{\cT}{\mathcal{T}}
\newcommand{\cU}{\mathcal{U}}
\newcommand{\cV}{\mathcal{V}}
\newcommand{\cW}{\mathcal{W}}
\newcommand{\cX}{\mathcal{X}}
\newcommand{\cY}{\mathcal{Y}}
\newcommand{\cZ}{\mathcal{Z}}


%%%%%%%%%%%%% mathematical fraktur  %%%%%%%%%%%%%%%%%%%%%
\newcommand{\mf}[1]{\mathfrak{#1}}

\newcommand{\fA}{\mathfrak{A}}
\newcommand{\fB}{\mathfrak{B}}
\newcommand{\fC}{\mathfrak{C}}
\newcommand{\fD}{\mathfrak{D}}
\newcommand{\fE}{\mathfrak{E}}
\newcommand{\fF}{\mathfrak{F}}
\newcommand{\fG}{\mathfrak{G}}
\newcommand{\fH}{\mathfrak{H}}
\newcommand{\fI}{\mathfrak{I}}
\newcommand{\fJ}{\mathfrak{J}}
\newcommand{\fK}{\mathfrak{K}}
\newcommand{\fL}{\mathfrak{L}}
\newcommand{\fM}{\mathfrak{M}}
\newcommand{\fN}{\mathfrak{N}}
\newcommand{\fO}{\mathfrak{O}}
\newcommand{\fP}{\mathfrak{P}}
\newcommand{\fQ}{\mathfrak{Q}}
\newcommand{\fR}{\mathfrak{R}}
\newcommand{\fS}{\mathfrak{S}}
\newcommand{\fT}{\mathfrak{T}}
\newcommand{\fU}{\mathfrak{U}}
\newcommand{\fV}{\mathfrak{V}}
\newcommand{\fW}{\mathfrak{W}}
\newcommand{\fX}{\mathfrak{X}}
\newcommand{\fY}{\mathfrak{Y}}
\newcommand{\fZ}{\mathfrak{Z}}


%%%%%%%%%%%%% mathematical script (euler)  %%%%%%%%%%%%%%%%%%%%%
\newcommand{\ms}[1]{\mathscr{#1}}

\newcommand{\sA}{\mathscr{A}}
\newcommand{\sB}{\mathscr{B}}
\newcommand{\sC}{\mathscr{C}}
\newcommand{\sD}{\mathscr{D}}
\newcommand{\sE}{\mathscr{E}}
\newcommand{\sF}{\mathscr{F}}
\newcommand{\sG}{\mathscr{G}}
\newcommand{\sH}{\mathscr{H}}
\newcommand{\sI}{\mathscr{I}}
\newcommand{\sJ}{\mathscr{J}}
\newcommand{\sK}{\mathscr{K}}
\newcommand{\sL}{\mathscr{L}}
\newcommand{\sM}{\mathscr{M}}
\newcommand{\sN}{\mathscr{N}}
\newcommand{\sO}{\mathscr{O}}
\newcommand{\sP}{\mathscr{P}}
\newcommand{\sQ}{\mathscr{Q}}
\newcommand{\sR}{\mathscr{R}}
\newcommand{\sS}{\mathscr{S}}
\newcommand{\sT}{\mathscr{T}}
\newcommand{\sU}{\mathscr{U}}
\newcommand{\sV}{\mathscr{V}}
\newcommand{\sW}{\mathscr{W}}
\newcommand{\sX}{\mathscr{X}}
\newcommand{\sY}{\mathscr{Y}}
\newcommand{\sZ}{\mathscr{Z}}


%%%%%%%%%%    Math operators    %%%%%%%%%%%%%%%%%%%%%%%%%%%

\renewcommand{\Re}{\mathop{\textnormal{Re}}}  % real part
\renewcommand{\Im}{\mathop{\textnormal{Im}}}  % imaginary part


%%%%%%%%%%%  FURTHER COMMANDS  %%%%%%%%%%%%%%%

\newcommand{\Id}{\mathrm{Id}}


%%%%%%%%%%%  STUDENT COMMANDS  %%%%%%%%%%%%%%%
%% Hier können Sie Ihre eigene LaTeX kommandos hinzufügen. %%
\newtheorem*{theorem*}{Theorem}
%% this allows for theorems which are not automatically numbered
\newtheorem{definition}{Definition}
\newtheorem{theorem}{Theorem}
\newtheorem{lemma}{Lemma}
\newtheorem{example}{Example}

%%%%%%%%%%%%%%%%%%%%%%%%%%%%%%%%%%%%%%%%%%%


\begin{document}
{\Large\bf Analysis 08}\\ \\
\hrule
%%%%%%%%%%%%%%%%%%%%%%%%%%%%%%%%%%%%%%%

\section*{Nr 3.1}
\subsection*{1}

Sei ein Wert $c>\limsup a_{n}$, sei ein $\varepsilon >0$, gewählt sodass $c-\varepsilon >\limsup a_{n}$\\
Nach der Definition des Limes superior gilt:
\begin{align*}
	         & \exists N: \forall m>N: a_{m}<c-\varepsilon \\
	\implies & |a_{m}-c|>\varepsilon, \forall m>N
\end{align*}
Keine Teilfolge von $(a_{n})$ kann somit gegen ein $c>\limsup a_{n}$ konvergieren und es kann keinen Häufungspunkt $c> \limsup a_{n}$ geben.\\

$\limsup a_{n}$ ist also eine obere Schranke von $H$.\\
Wie im Beweis zum Satz von Bolzano-Weierstrass gezeigt, gilt für beschränkte Folgen, dass es eine Teilfolge $k(n)$ von $(a_{n})_{n}$ gibt, sodass:
\begin{align*}
	|a_{k(n+1)}-b_{k_{n}+1}|<\frac{1}{n} \\
	b_{k(n-1)}-\frac{1}{n}\leq a_{k(n)}\leq b_{k(n)}
\end{align*}
Wobei $b_{n}=\sup \{a_{m}: m\geq n\}$ und $\lim \limits_{n\to\infty}^{""}b_{n}=\limsup a_{n}$\\
Nach dem Sandwichsatz konvergiert $a_{k(n)}$ gegen $\limsup a_{n}$.\\
$\limsup a_{n}$ ist somit ein Häufungspunkt und das Supremum von $H$.
\subsection*{2}

Nach Lemma 4.20 ist $b$ genau dann ein Häufungspunkt, wenn sich für alle $\varepsilon >0$ in  $B_{\varepsilon }(b)$ unendlich viele Elemente von $(a_{n})_n$ befinden.\\
Sei ein $\varepsilon >0$ gewählt.
Im $\frac{\varepsilon}{2} -$Ball um $b$ befinden sich unendlich viele Häufungspunkte $b_{n}$.\\
Sei der Häufungspunkt $b_{n_{0}}\in B_{\frac{\varepsilon }{2}}(b)$ mit dem kleinsten Abstand zu $b$ ausgewählt. Im $\frac{\varepsilon }{2}$- Ball um $b_{n_{0}}$ befinden sich unendlich viele Elemente der Folge $(a_{n})_{n}$.\\

Es gilt:\\
\begin{align*}
	 & b_{n_{0}}\in B_{\frac{\varepsilon }{2}}(b)                                                                             \\
	 & b-\frac{\varepsilon }{2}< b_{n_{0}}< b+\frac{\varepsilon }{2}                                                          \\
	 & b-\varepsilon < b_{n_{0}}- \frac{\varepsilon }{2}<b_{n_{0}}+\frac{\varepsilon }{2}<b+{\varepsilon }                    \\
	 & x\in (b_{n_{0}}-\frac{\varepsilon }{2}, b_{n_{0}}+\frac{\varepsilon }{2})\implies x\in (a-\varepsilon ,a+\varepsilon ) \\
	 & B_{\frac{\varepsilon }{2}}(b_{n_{0}})\subseteq B_{\varepsilon }(b)
\end{align*}
Da $B_{\frac{\varepsilon }{2}}(b_{n_{0}})\subseteq B_{{\varepsilon }}(b)$ befinden sich auch in  $B_{\varepsilon }(b)$ unendlich viele Elemente $a_{k}$ der Folge $(a_{n})_{n}$ und $b$ ist ein Häufungspunkt.

\section*{3.3}
1) Die Reihe $\sum_{n=1}^{\infty} \frac{x^{n}}{n!}$ konvergiert für alle $x\in \mathbb{R}$.\\
Somit konvergiert auch die Reihe $\sum_{n=1}^{\infty}\frac{(2x)^{n}}{n!}=\sum_{n=1}^{\infty}\left(\frac{2n}{n!}\right)x^{n}$.
\bigskip


2) $$\sqrt[n]{\frac{2^{n}}{n^{2}}}=2\cdot \left(\frac{1}{\sqrt[n]{n}}\right)^{2}$$
Die Folge $\left(\frac{1}{\sqrt[n]{n}}\right)^{2}$ konvergiert nach den Rechenregeln für konvergente Folgen gegen 1, da $\sqrt[n]{n}$ gegen 1 konvergiert.\\
Somit gilt:
\begin{align*}
	\limsup \left(\frac{1}{\sqrt[n]{n}}\right)^{2}=1 \\
	\limsup \sqrt[n]{\frac{2n}{n^{2}}}=2
\end{align*}
Nach Cauchy-Hardamard gilt: $$
	R= \frac{1}{2}
$$
\bigskip


3)
$$
	\sqrt[n]{n^{k}}=n^{\frac{k}{n}}=\left(\sqrt[n]{n}\right)^{k}
$$

Da $k$ ein konkretes Element der natürlichen Zahlen gilt
\begin{align*}
	\lim \limits_{n\to\infty}^{""}\left(\sqrt[n]{n}\right)^{k}= \lim \limits_{n\to \infty}^{""}\sqrt[n]{n}=1 \\
	\limsup \sqrt[n]{n^{k}}=1                                                                                \\
	\textnormal{Nach Cauchy-Hadamard: }R = 1
\end{align*}

\subsection*{4}
Sei $x\geq \theta >\sqrt{3}$.
\begin{align*}
	 & \limsup \sqrt[n]{3^{-n}\cdot n^{3}\cdot x^{2n}}      \\
	 & = \limsup 3^{-1}\cdot n^{\frac{1}{n}}^{3}\cdot x^{2} \\
	 & = \frac{1}{3}x^{2}                                   \\
	 & > \frac{1}{3}\theta^{2}>1
\end{align*}

Sei $x\leq \theta <\sqrt{3}$.
\begin{align*}
	 & \limsup \sqrt[n]{3^{-n}\cdot n^{3}\cdot x^{2n}}      \\
	 & = \limsup 3^{-1}\cdot n^{\frac{1}{n}}^{3}\cdot x^{2} \\
	 & = \frac{1}{3}x^{2}                                   \\
	 & > \frac{1}{3}\theta^{2}>1
\end{align*}

\section*{3.4}
Sei $(b_{n})_{n}$ eine Teilfolge von $a_{n}$, sodass $b_{n}\neq 0$.\\
Es gilt: $\sum_{n=0}^{\infty}a_{n}=\sum_{n=0}^{\infty}b_{n}$
\begin{align*}
	 & |b_{n}|\geq 1              \\
	%&\sqrt[n]{b_{n}}\geq 1\\
	%&\sqrt[n]{b_{n}}\cdot x^{n}\geq x^{n}\\
	 & b_{n}\cdot x^{n}\geq x^{n}
\end{align*}

Sei $x>1$.\\
$x^{n}$ ist keine Nullfolge, daher divergiert $\sum_{n=0}^{\infty}b_{n}x^{n}=\sum_{n=0}^{\infty}a_{n}x^{n}$
\section*{3.5}
\subsection*{Lemma 1}
Zu zeigen: Sei $\phi: \mathbb{N} \to A$ eine surjektive Abbildung. Es existiert dann eine Bijektion $\varphi: M\to A$, wobei $M=\{1,\dots ,N\}, N\in\mathbb{N} $ oder $M=\mathbb{N} $
\bigskip

Behauptung 1: \\
Zu zeigen ist, dass es eine Teilmenge $B\subseteq \mathbb{N}$, sodass es eine Bijektion $f: A\to B$ gibt.
\\Sei $f(a)= \min \{x\in \mathbb{N} : \phi (x)=a\}$
\bigskip

Behauptung 2: \\
Zu zeigen ist, dass es für jedes $B\subseteq \mathbb{N} $ entweder die Funktion $g:\{1,\dots ,N\}\to B$ eine Bijektion für ein $N\in \mathbb{N} $ ist, oder  die Funktion $g:\mathbb{N} \to B$ eine Bijektion ist.\\
Sei
\begin{align*}
	g(1) = \min \{B\} \\
	g(n+1) = \min \{B \backslash \{ g(\{1,\dots ,n\})\}\}
\end{align*}
Wenn B endlich, dann ist $\max \{B\}$ definiert und es gibt ein $N\in \mathbb{N} $, sodass $$
	g(N) = \min \{B\backslash g (\{1,\dots ,N-1\})\}= \max \{B\}
$$
$g:\{1,\dots ,N\}\to B$ ist dann dann bijektiv.\\
\bigskip

Sei $B$ hat nicht endlich viele Elemente. Zeige, dass $g: \mathbb{N} \to B$ injektiv und surjektiv.\medskip

Injektiv:  
\begin{align*}
	&\textnormal{Sei } n\neq m ,\textnormal{O.b.d.A }m>n\\
	& g(m)\in B \backslash \{g(\{1,\dots ,m-1\})\}\subseteq B\backslash \{g(n)\}\\ 
	& \implies g(m) \neq g(n)
\end{align*}
Surjektiv: 
\begin{align*}
	&\textnormal{Sei } b\in B \textnormal{. $b$ ist größer als $n<b$ Elemente aus $B$.}\\
	&b = g(n)
\end{align*}

Es gibt somit entweder eine Bijektion $\varphi = f\circ g: \{1,\dots ,N\}\to A$, oder eine Bijektion $\varphi = f \circ g: \mathbb{N} \to A$.
\subsection*{Lemma 2}
Die Abbildung $\psi: \mathbb{N}_0\times \mathbb{N}_0 \to \mathbb{Q}$ mit
\begin{align*}\psi(i,j)=
	\begin{cases}
		\frac{k}{j+1}, \textnormal{falls } i =2k, k\in \mathbb{N  } _{0} \\
		- \frac{k}{j+1}\textnormal{falls } i =2k+1, k\in \mathbb{N} _{0}
	\end{cases}
\end{align*}
ist surjektiv.

Sei $q\in \mathbb{Q} $.\\
Sei $q\geq 0$
\begin{align}
	 & q = \frac{n}{m}, n,m\in \mathbb{N}_{0} \\
	 & \textnormal{Sei } i=2n                 \\
	 & \textnormal{Sei } j=m-1
\end{align}
Sei $q<0$
Beweis analog.

\subsection*{Lemma 3}
Wie im Beweis zum Cauchy-Produkt gezeigt, gibt es eine Abzählung $\phi: \mathbb{N} _{0}\to \mathbb{N}_{0} \times  \mathbb{N}_{0}$


\subsection*{Beweis 3.5}
Nach Lemma 2 ist $\psi$ surjektiv, jedes $q\in \mathbb{Q} $ wird von einem $(i,j)\in \mathbb{N} _{0}\times  \mathbb{N} _{0}$ getroffen.
Jedes $(i,j)\in \mathbb{N} _{0}\times \mathbb{N} _{0}$ wird von einem $n\in \mathbb{N}$ getroffen.\\
Demnach wird jedes $q\in \mathbb{Q} $ von der Komposition $\psi \circ \phi: \mathbb{N} \to \mathbb{Q} $ getroffen, somit ist $\psi \circ \phi$ surjektiv.
\bigskip

Nach Lemma 1 gibt es somit eine Bijektion $\varphi :M \to \mathbb{Q}$, wobei $M=\{1,\dots ,N\}$, oder $M=\mathbb{N} $.
\bigskip

%Da $\{1,\dots ,N+1\}\subseteq \mathbb{Q} , \forall N\in \mathbb{N}$, kann es n
Da es keine Bijektion zwischen einer endlichen und einer unendlichen Menge geben kann, muss gelten $M=\mathbb{N} $. Somit gibt es eine Bijektion zwischen $\mathbb{Q}$ und $\mathbb{N} $. \qed
\\
\end{document}
