\documentclass{article}
\usepackage{graphicx} % Required for inserting images
\usepackage{amsmath}
\usepackage{amsthm}
\usepackage{amssymb}
\usepackage{stmaryrd}

\title{}
\author{
    Aaron Tsamaltoupis,
    Matr.Nr.: 3762396\\
}
\date{\today}

\newcommand{\ZZ}{\mathrm{Z\kern-.3em\raise-0.5ex\hbox{Z}}}

\begin{document}

\maketitle
\section*{Lemma1}
Beweise, dass die Folge $a_n$ konvergiert.

\section*{Beweis}
Nach Lemma 1 konvergiert a gegen einen Wert b.
Sei zum Zwecke des Widerspruches $a \neq b$.\\

Sei ein $\varepsilon < \frac{b-a}{4}$.\\
Es gibt demnach ein $M\in\mathbb{N} $, sodass $|b_{2n}-a|<\varepsilon , \forall n>M$.\\
Da $b_n$ gegen $a$ konvergiert und $a_n$ gegen $b$ konvergiert, kann ein $N>M$ gewählt werden, sodass
$$|b_{n}-a|<\frac{\varepsilon }{2} $$.\\
und
$$
	|a_n-b|<\frac{\varepsilon }{2}, \forall n>N
$$
Somit gilt: \\
$$
	b_n>a-\frac{\varepsilon }{2}
$$
und
$$
	a_n>b-\frac{\varepsilon }{2}
$$



Es gilt:
\begin{align}
	|b_{2n}-a| =
	 &  & \mid\frac{\sum^{2n}_{i=0}a_i}{2n}-a                \mid                                     \\
	 &  & \geq \mid\frac{\sum^{n}_{i=0}a_i}{2n}+\frac{\sum^{n}_{i=0}a_{n+i}}{2n}|-|a|                 \\
	 &  & >|\frac{a}{2}|+ \frac{n\cdot b}{2n}|-|\frac{\varepsilon }{4}|- |\frac{\varepsilon }{4}|-|a| \\
\end{align}

\newpage
\section*{Lemma1}
Zu zeigen ist, dass $a_n$ beschränkt ist.\\
Da $b_n$ konvergiert, ist $b_n$ nach Satz 4.5 beschränkt.\\
Es gilt: \\
\begin{align}
	\exists A: -A<b_n<A\in\mathbb{R} , \forall n\in\mathbb{N}             \\
	n\cdot (-A)<\sum^n_{i=1}a_i<n\cdot A                                  \\
	(n+1)\cdot (-A)<\sum^{n+1}_{i=1}a_i<(n+1)\cdot A                      \\
	-A = (n+1)(-A)-n(-A) < \sum^{n+1}_{i=1}a_i-\sum^{n}_{i=1}a_i= a_{n+1} \\
	A = (n+1)(A)-n(A) > \sum^{n+1}_{i=1}a_i-\sum^{n}_{i=1}a_i= a_{n+1}    \\
\end{align}
Somit gilt $-A<a_{n+1}<A, \forall n\in\mathbb{N} $ und $|a_n|$ ist beschränkt von $A$, bzw von $|a_1|$.

\section*{Beweis}
Da $a_n$ beschränkt ist, gibt es eine Teilfolge $d_n$ von $a_n$, die konvergiert.\\

\\

Es soll gezeigt werden, dass $a_n$ nur einen Häufungspunkt hat.\\
Seien $(a_{d_k})_k$ und $(a_{e_k})_k$ Teilfolgen von $(a_n)_n$, wobei$(a_{d_k})_k$ gegen $d$  und $(a_{e_k})_k$ gegen $e$ konvergiert.\\

Es seien alle Elemente $d_i$ der Folge $(a_{d_k})_k$ ausgewählt, für die gilt: \\
$$\exists m,a_{e_k}: a_m\leq d_i\leq a_{2m}, a_m\leq a_{e_k} \leq a_{2m}$$\\
Die Punkte $d_i$ bilden dabei eine Teilfolge von $a_n$, da $m$ beliebig groß gewählt werden kann.\\
Da






Die Menge aller Häufungspunkte von $a_n$ beträgt also 1.\\
Somit konvergiert $a_n$. Speziell muss $a_n$ gegen $a$ konvergieren. Würde $a_n$ gegen einen Wert $b\neq a$ konvergieren, würde nach nr 3.2 auf Blatt 04 auch die Folge $b_n$ gegen diesen Wert konvergieren. Dies ist nicht der Fall.






\end{document}

