\documentclass[a4paper,oneside,11pt]{article}
\usepackage[utf8]{inputenc}

\usepackage[ngerman]{babel} % For correct hyphenation

\usepackage{mathtools}
\usepackage{amssymb,amsmath,amsthm}
\usepackage[mathscr]{eucal}

\usepackage[textwidth=17cm,top=1.5cm,bottom=1.5cm,nohead]{geometry}

\setlength{\parindent}{0mm} % no paragraph indentation


%%%%%%%%%%%%%%%%%%%%%%%%%%%%%%%%%%%%%%%%%%%%%%%%%%%%%%%

% Abbreviations

%%%%%%%%%%%%%%%%%%%%%%%%%%%%%%%%%%%%%%%%%%%%%%%%%%%%%%%
% single letters in different fonts 

%%%%%%%%%%%% mathematical bold  %%%%%%%%%%%%%%%%%%%%

\newcommand{\bA}{\mathbb{A}}
\newcommand{\bB}{\mathbb{B}}
\newcommand{\bC}{\mathbb{C}}
\newcommand{\bD}{\mathbb{D}}
\newcommand{\bE}{\mathbb{E}}
\newcommand{\bF}{\mathbb{F}}
\newcommand{\bG}{\mathbb{G}}
\newcommand{\bH}{\mathbb{H}}
\newcommand{\bI}{\mathbb{I}}
\newcommand{\bJ}{\mathbb{J}}
\newcommand{\bK}{\mathbb{K}}
\newcommand{\bL}{\mathbb{L}}
\newcommand{\bM}{\mathbb{M}}
\newcommand{\bN}{\mathbb{N}}
\newcommand{\bO}{\mathbb{O}}
\newcommand{\bP}{\mathbb{P}}
\newcommand{\bQ}{\mathbb{Q}}
\newcommand{\bR}{\mathbb{R}}
\newcommand{\bS}{\mathbb{S}}
\newcommand{\bT}{\mathbb{T}}
\newcommand{\bU}{\mathbb{U}}
\newcommand{\bV}{\mathbb{V}}
\newcommand{\bW}{\mathbb{W}}
\newcommand{\bX}{\mathbb{X}}
\newcommand{\bY}{\mathbb{Y}}
\newcommand{\bZ}{\mathbb{Z}}


%%%%%%%%% calligraphic %%%%%%%%%%%%%%%%%%%%%%%
\newcommand{\mc}[1]{\mathcal{#1}}

\newcommand{\cA}{\mathcal{A}}
\newcommand{\cB}{\mathcal{B}}
\newcommand{\cC}{\mathcal{C}}
\newcommand{\cD}{\mathcal{D}}
\newcommand{\cE}{\mathcal{E}}
\newcommand{\cF}{\mathcal{F}}
\newcommand{\cG}{\mathcal{G}}
\newcommand{\cH}{\mathcal{H}}
\newcommand{\cI}{\mathcal{I}}
\newcommand{\cJ}{\mathcal{J}}
\newcommand{\cK}{\mathcal{K}}
\newcommand{\cL}{\mathcal{L}}
\newcommand{\cM}{\mathcal{M}}
\newcommand{\cN}{\mathcal{N}}
\newcommand{\cO}{\mathcal{O}}
\newcommand{\cP}{\mathcal{P}}
\newcommand{\cQ}{\mathcal{Q}}
\newcommand{\cR}{\mathcal{R}}
\newcommand{\cS}{\mathcal{S}}
\newcommand{\cT}{\mathcal{T}}
\newcommand{\cU}{\mathcal{U}}
\newcommand{\cV}{\mathcal{V}}
\newcommand{\cW}{\mathcal{W}}
\newcommand{\cX}{\mathcal{X}}
\newcommand{\cY}{\mathcal{Y}}
\newcommand{\cZ}{\mathcal{Z}}


%%%%%%%%%%%%% mathematical fraktur  %%%%%%%%%%%%%%%%%%%%%
\newcommand{\mf}[1]{\mathfrak{#1}}

\newcommand{\fA}{\mathfrak{A}}
\newcommand{\fB}{\mathfrak{B}}
\newcommand{\fC}{\mathfrak{C}}
\newcommand{\fD}{\mathfrak{D}}
\newcommand{\fE}{\mathfrak{E}}
\newcommand{\fF}{\mathfrak{F}}
\newcommand{\fG}{\mathfrak{G}}
\newcommand{\fH}{\mathfrak{H}}
\newcommand{\fI}{\mathfrak{I}}
\newcommand{\fJ}{\mathfrak{J}}
\newcommand{\fK}{\mathfrak{K}}
\newcommand{\fL}{\mathfrak{L}}
\newcommand{\fM}{\mathfrak{M}}
\newcommand{\fN}{\mathfrak{N}}
\newcommand{\fO}{\mathfrak{O}}
\newcommand{\fP}{\mathfrak{P}}
\newcommand{\fQ}{\mathfrak{Q}}
\newcommand{\fR}{\mathfrak{R}}
\newcommand{\fS}{\mathfrak{S}}
\newcommand{\fT}{\mathfrak{T}}
\newcommand{\fU}{\mathfrak{U}}
\newcommand{\fV}{\mathfrak{V}}
\newcommand{\fW}{\mathfrak{W}}
\newcommand{\fX}{\mathfrak{X}}
\newcommand{\fY}{\mathfrak{Y}}
\newcommand{\fZ}{\mathfrak{Z}}


%%%%%%%%%%%%% mathematical script (euler)  %%%%%%%%%%%%%%%%%%%%%
\newcommand{\ms}[1]{\mathscr{#1}}

\newcommand{\sA}{\mathscr{A}}
\newcommand{\sB}{\mathscr{B}}
\newcommand{\sC}{\mathscr{C}}
\newcommand{\sD}{\mathscr{D}}
\newcommand{\sE}{\mathscr{E}}
\newcommand{\sF}{\mathscr{F}}
\newcommand{\sG}{\mathscr{G}}
\newcommand{\sH}{\mathscr{H}}
\newcommand{\sI}{\mathscr{I}}
\newcommand{\sJ}{\mathscr{J}}
\newcommand{\sK}{\mathscr{K}}
\newcommand{\sL}{\mathscr{L}}
\newcommand{\sM}{\mathscr{M}}
\newcommand{\sN}{\mathscr{N}}
\newcommand{\sO}{\mathscr{O}}
\newcommand{\sP}{\mathscr{P}}
\newcommand{\sQ}{\mathscr{Q}}
\newcommand{\sR}{\mathscr{R}}
\newcommand{\sS}{\mathscr{S}}
\newcommand{\sT}{\mathscr{T}}
\newcommand{\sU}{\mathscr{U}}
\newcommand{\sV}{\mathscr{V}}
\newcommand{\sW}{\mathscr{W}}
\newcommand{\sX}{\mathscr{X}}
\newcommand{\sY}{\mathscr{Y}}
\newcommand{\sZ}{\mathscr{Z}}


%%%%%%%%%%    Math operators    %%%%%%%%%%%%%%%%%%%%%%%%%%%

\renewcommand{\Re}{\mathop{\textnormal{Re}}}  % real part
\renewcommand{\Im}{\mathop{\textnormal{Im}}}  % imaginary part


%%%%%%%%%%%  FURTHER COMMANDS  %%%%%%%%%%%%%%%

\newcommand{\Id}{\mathrm{Id}}


%%%%%%%%%%%  STUDENT COMMANDS  %%%%%%%%%%%%%%%
%% Hier können Sie Ihre eigene LaTeX kommandos hinzufügen. %%
\newtheorem*{theorem*}{Theorem}
%% this allows for theorems which are not automatically numbered
\newtheorem{definition}{Definition}
\newtheorem{theorem}{Theorem}
\newtheorem{lemma}{Lemma}
\newtheorem{example}{Example}

%%%%%%%%%%%%%%%%%%%%%%%%%%%%%%%%%%%%%%%%%%%


\begin{document}
{\Large\bf Blatt 09}\\  tags:\  {}\\
\hrule
%%%%%%%%%%%%%%%%%%%%%%%%%%%%%%%%%%%%%%%
\bigskip

\section*{Nr 3.4}
\subsection*{1}
Injektiv: \bigskip
Sei $x,y\in [0,+\infty [, f(x)=f(y)$\medskip

$x^{2}=y^{2}$ \\
$x=y$

\bigskip

Surjektiv:\\
Für jedes $x\in \mathbb{R}$ existiert ein $c^{2}=x, c\in \mathbb{R} $\\
$[0,\infty [\subseteq \mathbb{R} $
\subsection*{2}
Nicht injektiv:\medskip

$-1 \neq 1$\\
$f_{2}(-1)=(-1)^{4}=1=1^{4}=f_{2}(1)$
\bigskip

Nicht surjektiv: \\
Sei $-1=f_{2}(x)=x^{4}$\\
$-1=(x^{2})^{2}=y^{2}$.
Dies ist ein Widerspruch, da die Gleichung $y^{2}+1=0$ keine Lösung in den reellen Zahlen hat.$-1$ ist also kein Funktionswert eines $x\in \mathbb{R}$
\subsection*{3}
Injektiv:\medskip

Sei $f(x),f(y)\in \mathbb{N}\cap ]3,\infty [$
			\begin{align*}
				x+3=y+3 \\
				x=y
			\end{align*}
			\medskip

			Surjektiv:\medskip

			Sei $x\in \mathbb{N} \cap ]3,\infty [$
\begin{align*}
	x-3>1             \\
	x-3\in \mathbb{N} \\
	f(x-3) = x-3+3 = x
\end{align*}
\subsection*{4)}
Nicht injektiv: \medskip
\begin{align*}
	(2,3)\neq (3,2) \\
	2+3=3+2
\end{align*}
Surjektiv: \medskip

Sei $x\in \mathbb{R} \backslash \{ 0\}$\\
$$
	x=f_{4}(\frac{x}{2},\frac{x}{2})=\frac{x}{2}+\frac{x}{2}
$$
Sei $x=0$\\
$$
	x=f_{4}(1,-1)=1-1
$$
\subsection*{5)}
Nicht surjektiv:\medskip

\begin{align*}
	1\in ]0,\infty[      \\
	f_{5}(x)=\exp(x)+1=1 \\
	\iff \exp(x)= 0
\end{align*}
Widerspruch zu (i), demnach kann es kein $x\in \mathbb{R}$ geben, sodass $f_{5}=1$.
\medskip

Bijektiv: \medskip

Sei $x\neq y, x,y\in \mathbb{R} $, oBdA: $x<y$ .
\begin{align}
	 & x^{n}<y^{n}, \forall n< 1                                                 \\
	 & \sum_{n=0}^{\infty}\frac{1}{n!}x^{n}<\sum_{n=0}^{\infty}\frac{1}{n!}y^{n} \\
	 & \exp(x)<\exp(y)                                                           \\
	 & \exp(x)\neq \exp(y)
\end{align}
\section*{3.5}
Sei $b_{n}$ folgende Folge:
\begin{align}
	b_{n} =	 \lim \limits_{M \to\infty} \sum_{k=0}^{M}a_{n+k} \binom{n+k}{k}\cdot z_{0}^{k}
\end{align}
\subsection*{Lemma 1}
\begin{align}
	\forall N\in \mathbb{N}, \forall z\in B_{z}(r) :\sum_{n=0}^{N}a_{n}(z+z_{0})^{n}=\sum_{n=0}^{N}z_{n}\sum_{k=0}^{N-n}a_{n+k}\binom{n+k}{k}\cdot z_{0}^{k}
\end{align}

\subsection*{Lemma 2}
\begin{align*}
	2\cdot \sum_{n=0}^{N}z_{n}\sum_{k=0}^{N-n}a_{n+k}\binom{n+k}{k}\cdot z_{0}^{k}=\sum_{n=0}^{N}z^{n} \sum_{n=0}^{N}a_{n+k}\binom{n+k}{k}\cdot z_{0}^{k}, \forall N\in \mathbb{N},\forall z\in B_{r}(z_{0})
\end{align*}
\subsection*{Beweis}
Es gilt
Nach Lemma 1 und Lemma 2 für alle $z\in B_{r}(z_{0})$
\begin{align*}
	 & \sum_{n=0}^{N}a_{n}(z+z_{0})^{n}=\sum_{n=0}^{N}z^{n}\cdot \frac{1}{2}\cdot \sum_{n=0}^{N}a_{n+k}\binom{n+k}{k}\cdot z_{0}^{k}                               \\
	 & \lim \limits_{N\to\infty }
	\sum_{n=0}^{N}a_{n}(z+z_{0})^{n}=\lim \limits_{N \to\infty}\sum_{n=0}^{N}z^{n}\cdot \frac{1}{2}\cdot \sum_{n=0}^{N}a_{n+k}\binom{n+k}{k}\cdot z_{0}^{k}        \\
	 & P(z+z_{0})=\lim \limits_{N \to\infty}\sum_{n=0}^{N}z^{n}\cdot \frac{1}{2}\cdot \lim \limits_{M \to\infty}\sum_{n=0}^{M}a_{n+k}\binom{n+k}{k}\cdot z_{0}^{k} \\
	 & P(z+z_{0})=\sum_{n=0}^{\infty }z^{n}\cdot b_{n}=Q(z)                                                                                                        \\
\end{align*}

\section*{3.5}
\begin{align*}
	P(z+z_{0})=\sum_{n=0}^{\infty}a_{n}(z+z_{0})^{n} = \sum_{n=0}^{\infty}a_{n}\cdot \sum_{k=0}^{n}\binom{n}{k}z^{n-k}z_{0}^{k}=\sum_{n=0}^{\infty}\cdot \sum_{k=0}^{n}z^{n-k}a_{n}\binom{n}{k}z_{0}^{k}
\end{align*}

Es seien alle Summanden der Summe betrachtet, die den speziellen Faktor $z^{i}, i=n-k$ haben.
\begin{align*}
	z^{n-k}\cdot a_{n}\cdot z_{0}^{k}\binom{n}{k} \\
	\textnormal{Sei $i=n-k$}                      \\
	z^{i}a_{i+k}\cdot z_{0}^{k}\binom{i+k}{k}
\end{align*}

Die Summe alle Summanden der Summe $P(z+z_{0})$, die den Faktor $z^{i}$ haben kann wie folgt geschrieben werden:\begin{align*}
	S_{i}=z^{i}\cdot \sum_{k=0}^{\infty}a_{i+k}\binom{i+k}{k}\cdot z_{0}^{k}
\end{align*}


Da jeder Summand von $P(z+z_{0})$ den Faktor $z^{i}$ für ein $i\in \mathbb{N} $ enthält, kann $P(z+z_{0})^{n}$ auch so geschrieben werden:
\begin{align*}
	P(z+z_{0}) & =\sum_{n=0}^{\infty}S_{n}                                                                 \\
	           & = \sum_{n=0}^{\infty}z^{n}\cdot \sum_{k=0}^{\infty}a_{n+k}\binom{n+{k}}{k}\cdot z_{0}^{k} \\
	           & = \sum_{n=0}^{\infty}z^{n}\cdot b_{n}=Q(z)
\end{align*}
Dies gilt für alle $z\in \mathbb{C}$, sodass $(z+z_{0})\in B_{R}(0)$.\\
Die Aussage gilt demnach für alle $z$,sodass $z\in B_{R-|z_{0}|}(0)=B_{r}(0)$
\qed
\end{document}

