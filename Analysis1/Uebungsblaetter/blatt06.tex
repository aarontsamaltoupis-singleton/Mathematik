\documentclass{article}
\usepackage{graphicx} % Required for inserting images
\usepackage{amsmath}
\usepackage{amsthm}
\usepackage{amssymb}
\usepackage{stmaryrd}


\title{}
\author{
    Aaron Tsamaltoupis,
    Matr.Nr.: 3762396\\
}
\date{\today}

\newcommand{\ZZ}{\mathrm{Z\kern-.3em\raise-0.5ex\hbox{Z}}}

\begin{document}

\maketitle
\section*{Nr 3.2}
Die Summe $\sum_{n=1}^{\infty}$ konvergiert
\newpage

\section*{Nr 3.3}
Zu zeigen ist, dass
$$
	\sum_{n}n(a_n-a_{n+1}) = \sum_{n}a_n - n\cdot a_{n+1}
$$
\textbf{Induktionsanfang}\\
$$
	\sum_{n=1}^1= 1(a_1-a_2) = a_1 - 1\cdot a_2
$$
\textbf{Induktionsschritt}\\
Sei $$
	\sum_{n=1}^m m(a_m-a_{m+1}) = \sum_{n=1}^m a_m - m\cdot a_{m+1}
$$
Aufgrund der notwendigen Bedinung der Konvergenz von Reihen muss $a_n$ gegen 0 konvergieren.\\
Da $a_n$ monoton ist und gegen 0 konvergiert, besteht die Folge entweder nur aus positiven, oder nur aus negativen Werten. Demnach gilt:
$$
	|\sum_{n}a_n - n\cdot a_{n+1}|<|\sum_{n}a_n|
$$
$\sum_{n}a_n$ konvergiert außerdem absolut, demnach konvergiert auch $|\sum_{n}a_n|$.
Nach der Monotonie der Grenzwerte konvergiert $|\sum_{n}n(a_n-a_{n+1})|$ zu einem Wert kleiner als der Grenzwert von $|\sum_n a_n|$. \\
Somit konvergiert auch  die Folge $$\sum_n n(a_{n+1} -a_n)$$.
\end{document}

