\documentclass[a4paper,oneside,11pt]{article}
\usepackage[utf8]{inputenc}

\usepackage[ngerman]{babel} % For correct hyphenation

\usepackage{mathtools}
\usepackage{amssymb,amsmath,amsthm}
\usepackage[mathscr]{eucal}

\usepackage[textwidth=17cm,top=1.5cm,bottom=1.5cm,nohead]{geometry}

\setlength{\parindent}{0mm} % no paragraph indentation


%%%%%%%%%%%%%%%%%%%%%%%%%%%%%%%%%%%%%%%%%%%%%%%%%%%%%%%

% Abbreviations

%%%%%%%%%%%%%%%%%%%%%%%%%%%%%%%%%%%%%%%%%%%%%%%%%%%%%%%
% single letters in different fonts 

%%%%%%%%%%%% mathematical bold  %%%%%%%%%%%%%%%%%%%%

\newcommand{\bA}{\mathbb{A}}
\newcommand{\bB}{\mathbb{B}}
\newcommand{\bC}{\mathbb{C}}
\newcommand{\bD}{\mathbb{D}}
\newcommand{\bE}{\mathbb{E}}
\newcommand{\bF}{\mathbb{F}}
\newcommand{\bG}{\mathbb{G}}
\newcommand{\bH}{\mathbb{H}}
\newcommand{\bI}{\mathbb{I}}
\newcommand{\bJ}{\mathbb{J}}
\newcommand{\bK}{\mathbb{K}}
\newcommand{\bL}{\mathbb{L}}
\newcommand{\bM}{\mathbb{M}}
\newcommand{\bN}{\mathbb{N}}
\newcommand{\bO}{\mathbb{O}}
\newcommand{\bP}{\mathbb{P}}
\newcommand{\bQ}{\mathbb{Q}}
\newcommand{\bR}{\mathbb{R}}
\newcommand{\bS}{\mathbb{S}}
\newcommand{\bT}{\mathbb{T}}
\newcommand{\bU}{\mathbb{U}}
\newcommand{\bV}{\mathbb{V}}
\newcommand{\bW}{\mathbb{W}}
\newcommand{\bX}{\mathbb{X}}
\newcommand{\bY}{\mathbb{Y}}
\newcommand{\bZ}{\mathbb{Z}}


%%%%%%%%% calligraphic %%%%%%%%%%%%%%%%%%%%%%%
\newcommand{\mc}[1]{\mathcal{#1}}

\newcommand{\cA}{\mathcal{A}}
\newcommand{\cB}{\mathcal{B}}
\newcommand{\cC}{\mathcal{C}}
\newcommand{\cD}{\mathcal{D}}
\newcommand{\cE}{\mathcal{E}}
\newcommand{\cF}{\mathcal{F}}
\newcommand{\cG}{\mathcal{G}}
\newcommand{\cH}{\mathcal{H}}
\newcommand{\cI}{\mathcal{I}}
\newcommand{\cJ}{\mathcal{J}}
\newcommand{\cK}{\mathcal{K}}
\newcommand{\cL}{\mathcal{L}}
\newcommand{\cM}{\mathcal{M}}
\newcommand{\cN}{\mathcal{N}}
\newcommand{\cO}{\mathcal{O}}
\newcommand{\cP}{\mathcal{P}}
\newcommand{\cQ}{\mathcal{Q}}
\newcommand{\cR}{\mathcal{R}}
\newcommand{\cS}{\mathcal{S}}
\newcommand{\cT}{\mathcal{T}}
\newcommand{\cU}{\mathcal{U}}
\newcommand{\cV}{\mathcal{V}}
\newcommand{\cW}{\mathcal{W}}
\newcommand{\cX}{\mathcal{X}}
\newcommand{\cY}{\mathcal{Y}}
\newcommand{\cZ}{\mathcal{Z}}


%%%%%%%%%%%%% mathematical fraktur  %%%%%%%%%%%%%%%%%%%%%
\newcommand{\mf}[1]{\mathfrak{#1}}

\newcommand{\fA}{\mathfrak{A}}
\newcommand{\fB}{\mathfrak{B}}
\newcommand{\fC}{\mathfrak{C}}
\newcommand{\fD}{\mathfrak{D}}
\newcommand{\fE}{\mathfrak{E}}
\newcommand{\fF}{\mathfrak{F}}
\newcommand{\fG}{\mathfrak{G}}
\newcommand{\fH}{\mathfrak{H}}
\newcommand{\fI}{\mathfrak{I}}
\newcommand{\fJ}{\mathfrak{J}}
\newcommand{\fK}{\mathfrak{K}}
\newcommand{\fL}{\mathfrak{L}}
\newcommand{\fM}{\mathfrak{M}}
\newcommand{\fN}{\mathfrak{N}}
\newcommand{\fO}{\mathfrak{O}}
\newcommand{\fP}{\mathfrak{P}}
\newcommand{\fQ}{\mathfrak{Q}}
\newcommand{\fR}{\mathfrak{R}}
\newcommand{\fS}{\mathfrak{S}}
\newcommand{\fT}{\mathfrak{T}}
\newcommand{\fU}{\mathfrak{U}}
\newcommand{\fV}{\mathfrak{V}}
\newcommand{\fW}{\mathfrak{W}}
\newcommand{\fX}{\mathfrak{X}}
\newcommand{\fY}{\mathfrak{Y}}
\newcommand{\fZ}{\mathfrak{Z}}


%%%%%%%%%%%%% mathematical script (euler)  %%%%%%%%%%%%%%%%%%%%%
\newcommand{\ms}[1]{\mathscr{#1}}

\newcommand{\sA}{\mathscr{A}}
\newcommand{\sB}{\mathscr{B}}
\newcommand{\sC}{\mathscr{C}}
\newcommand{\sD}{\mathscr{D}}
\newcommand{\sE}{\mathscr{E}}
\newcommand{\sF}{\mathscr{F}}
\newcommand{\sG}{\mathscr{G}}
\newcommand{\sH}{\mathscr{H}}
\newcommand{\sI}{\mathscr{I}}
\newcommand{\sJ}{\mathscr{J}}
\newcommand{\sK}{\mathscr{K}}
\newcommand{\sL}{\mathscr{L}}
\newcommand{\sM}{\mathscr{M}}
\newcommand{\sN}{\mathscr{N}}
\newcommand{\sO}{\mathscr{O}}
\newcommand{\sP}{\mathscr{P}}
\newcommand{\sQ}{\mathscr{Q}}
\newcommand{\sR}{\mathscr{R}}
\newcommand{\sS}{\mathscr{S}}
\newcommand{\sT}{\mathscr{T}}
\newcommand{\sU}{\mathscr{U}}
\newcommand{\sV}{\mathscr{V}}
\newcommand{\sW}{\mathscr{W}}
\newcommand{\sX}{\mathscr{X}}
\newcommand{\sY}{\mathscr{Y}}
\newcommand{\sZ}{\mathscr{Z}}


%%%%%%%%%%    Math operators    %%%%%%%%%%%%%%%%%%%%%%%%%%%

\renewcommand{\Re}{\mathop{\textnormal{Re}}}  % real part
\renewcommand{\Im}{\mathop{\textnormal{Im}}}  % imaginary part


%%%%%%%%%%%  FURTHER COMMANDS  %%%%%%%%%%%%%%%

\newcommand{\Id}{\mathrm{Id}}


%%%%%%%%%%%  STUDENT COMMANDS  %%%%%%%%%%%%%%%
%% Hier können Sie Ihre eigene LaTeX kommandos hinzufügen. %%
\newtheorem*{theorem*}{Theorem}
%% this allows for theorems which are not automatically numbered
\newtheorem{definition}{Definition}
\newtheorem{theorem}{Theorem}
\newtheorem{lemma}{Lemma}
\newtheorem{example}{Example}

%%%%%%%%%%%%%%%%%%%%%%%%%%%%%%%%%%%%%%%%%%%


\begin{document}

\begin{center}
	{\Large\bf Aufgaben für Analysis I} \\
	\medskip
	\textbf{\large Blatt NUMMER} \\
	\bigskip
	\textbf{IHRE NAME}
\end{center}

\bigskip
\hrule
\bigskip\bigskip

%%%%%%%%%%%%%%%%%%%%%%%%%%%%%%%%%%%%%%%

\section*{Aufgabe 3.1}
\subsection*{8)}
$h_n=n^k\cdot x^n$, wobei $|x|<1$\\

Es soll durch Induktion über n gezeigt werden, dass $\forall k\in\mathbb{N}:\exists N\in \mathbb{N}: x^n<\frac{1}{n^k},\forall n<N$.
\subsubsection*{Induktionsanfang}
Sei $n>k$, Sei $n$ so gewählt, dass $(\frac{n}{n+1})^k>x$.\\
Sei $x^n>\frac{1}{n^k}$. Es soll gezeigt werden, dass dann für ein späteres $k\cdot n$ die Ungleichung gilt.\\
Es gilt demnach:  $x^n=\frac{1}{n^k}+\varepsilon, \varepsilon \in\mathbb{R} $. Für größere n, kann $\varepsilon$ beliebig klein gewählt werden, da sowohl $x^n$, als auch $\frac{1}{n^k}$ gegen 0 konvergieren.\\
\\
\begin{align}
	x^{kn}= & (\frac{1}{n^k}+\varepsilon )^{k}                                    \\
	        & =\sum^{k}_{i=0}\binom{k}{i}\frac{1}{n^{2k-i}}\cdot \varepsilon ^{i} \\
	        & <\frac{1}{n^{2k}}                                                   \\
	        & <\frac{1}{n^{k}\cdot n^k}                                           \\
	        & <\frac{1}{(kn)^k }                                                  \\
\end{align}
Somit gilt $x^{nk}<\frac{1}{nk}$ als Induktionsanfang. Es soll durch den Induktionsschritt gezeigt werden, dass die Ungleichung $x^m<\frac{1}{m}$ ab diesem Punkt für alle $m>nk$ gilt.


\subsubsection*{Induktionsschritt}
Die Folge $(\frac{n}{n+1})$ konvergiert gegen 1.\\
Somit konvergiert auch die Folge $(\frac{n}{n+1})^{k}$\\
Sei also $n$ so gewählt, dass $(\frac{n}{n+1})^{k}>x$. Dies ist moeglich, da $x<1$.\\
\textbf{Induktionshypothese:} Sei $x^n<\frac{1}{n^k}$.
\begin{align}
	x^{n+1} & = x\cdot x^n                                          \\
	        & < (\frac{n}{n+1})^{k}\cdot x^n                        \\
	        & \overset{IH}{<}(\frac{n}{n+1})^{k}\cdot \frac{1}{n^k} \\
	        & = \frac{1}{(n+1)^k}
\end{align}
\\\\
Da gilt $x^n<\frac{1}{n^k	}$, gilt auch:
$x^{n}\cdot n^{k-1}<\frac{1}{n}, \forall n<N$.\\
Nach Beispielaufgabe $e7$ konvergiert $\frac{1}{n}$ gegen 0. Da $h(n)>0, \forall n\in\mathbb{N}$ konvergiert die Folge nach Satz 4.8 gegen 0.



\newpage
\section*{Aufgabe 3.2}

Sei ein $\varepsilon<0$.\\
Sei ein $n_0$ gewählt, sodass $|a-a_{n_0}|< \varepsilon $\\


Sei ein k gewählt, sodass $\frac{\max\{a_1,\dots ,a_{n_0}, n_0\cdot a+n_0\cdot  \varepsilon\cdot  b\}}{kn_0}<\frac{\varepsilon -|a-a_{n_0}|}{2}$
\\
Sei $N>n_0\cdot k$

\begin{align}
	|\frac{a_1+\dots +a_n}{n}|= |\frac{a_1+\dots +a_{n_0}}{n}+\frac{a_{n_0+1}+\dots +a_{2n_0}}{n}|
\end{align}
\newpage
\subsection*{Nr 3.4}
sei $k_0 = k^2$\\
Zu zeigen: $$k_0(1+1)^n>(1+q)^{n-1}+2(1+q)^{n-2}\dots +n$$
$$\iff k> d_n = \frac{1}{k(1+q)}+\frac{2}{k(1+q)^2}+\dots +\frac{n}{k(1+q)^2}$$
\\Sei $n_0$ so gewählt, dass $\frac{n_0}{k(q+1)^{n_0}}<\varepsilon $\\
Dies ist möglich, da $\frac{n_0}{(q+1)^{n_0}}$ gegen 0 konvergiert.\\
Es gibt somit eine Konstante $M$, sodass $$d_n=M  + \sum_{i=0}^{\infty}e_n$$, wobei
$$e_1<\varepsilon $$ und
$$e_{n+1} = \frac{e_{n}}{(q+1)k}+ \frac{e_n}{(q+1)k\cdot n_0} $$\\
Sei k nun so gewählt, dass $k(q+1)>4$.
Es gilt:
$e_n <  \frac{\varepsilon }{2^{n-1}} $\\
**Beweis fehlt**
\\
Es gilt somit:
$$
	\sum^{\infty}_{i=0} e_n < \sum^{\infty}_{i=0} \varepsilon \cdot \frac{1}{2^{n-1}} = m
$$

Somit gilt:
$$
	d_n < M+m\\
$$

k kann beliebig größer gewählt werden, sodass gilt $k>d_n$.\\
Die einzige Bedingung für k, dass $k(q+1)>4$ ist, bleibt dadurch bestehen.\\
Somit kann $k$ so gewählt werden, dass
$$
	k>d_n \\
	\iff k^2 = k_0 >d_n \cdot {k}\\
	=

$$
\newpage
\section{Lemma 1}
Sei k so gewählt, dass $k>\frac{1}{q-1}$.
\begin{align}
	e_1 = \frac{1}{q+1}                                                                                                          \\
	\textnormal{Die Folge} S' = e_1+\frac{2e_1}{q+1}+\frac{2^2e_1}{(q+1)^2}+\dots \textnormal{konvergiert gegen } \frac{1}{q-1}. \\
	\textnormal{Es gilt: }k>S'>S                                                                                                 \\
	\textnormal{Somit gilt: } k(1+q)^n>(1+q)^{n-1}+2(1+q)^{n-2}+\dots +n
\end{align}




\section*{Beweis}
Es soll gezeigt werden, dass $a_n< (\frac{c}{c+\delta})^n$ ,für ein $\delta>0$.
Da geizeigt wurde, dass $(\frac{c}{c+\delta})^n$ gegen 0 konvergiert, konvergiert aufgrund des Sandwich Satzes auch $a_n$ gegen 0.

Es gilt nach Lemma1:$$
	$a_n<c^{\sum^{n-1}_{i=1}i(q+1)^{(n-i)}}\cdot A^{(q+1)^n}$
$$

Sei $A=\frac{1}{(c+\delta)^k}$.\\
Somit gilt: $$
	a_n<\frac{c^{\sum^{n-1}_{i=1}i(q+1)^{(n-i)}}
	}{(c+\delta)^{k(q+1)^n}}=\frac{c^{c'}}{(c+\delta)^{a'}}$$

Nach Lemma 2 kann für $k$ ein Wert gewählt werden, sodass

$$\sum^{n-1}_{i=1}i(q+1)^{(n-i)}< k(q+1)^n$$, also
$$c'<a'$$

\\
Somit gilt
$$
	a_n<\frac{c^{c'}}{(c+\delta)^{a'}}<\frac{c^c'}{c}
$$

\newpage
\section*{Lemma1}
Zu zeigen ist, dass die Folge $$c'_n = \sum^{}$$


\section*{Beweis}
$a_n<c^{c_n'}\cdot A^{(1+q)^n}$
\\
Nach Lemma 1 ist gezeigt, dass der Exponent von $c$ $c_n'$ gegen einen Wert $l$ konvergiert.
\\



\\
Sei A eine beliebige reelle Zahl,  sodass $A < \frac{1}{c^l}$.
$$
	a_n<c^l\cdot A^{(1+q)^n}<\frac{c^l}{c^{l(1+q)^n}}\leq \frac{1}{c^{(1+q)^n}}< \frac{1}{c^n}
$$
Da $c>1$ konergiert $\frac{1}{c^n}$ gegen 0. Nach dem Sandwich Satz strebt $a_n$ somit gegen 0.\\
Da A beliebig war, gilt dies für alle $a_n$, für die gilt $a_1<A$.

\end{document}







