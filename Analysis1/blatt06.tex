\documentclass{article}
\usepackage{graphicx} % Required for inserting images
\usepackage{amsmath}
\usepackage{amsthm}
\usepackage{amssymb}
\usepackage{stmaryrd}


\title{}
\author{
    Aaron Tsamaltoupis,
    Matr.Nr.: 3762396\\
}
\date{\today}

\newcommand{\ZZ}{\mathrm{Z\kern-.3em\raise-0.5ex\hbox{Z}}}

\begin{document}

\maketitle
\section*{Nr 3.3}
$$
	\sum^{\infty}_{n=1}n(a_n-a_{n+1}) =\lim \limits_{n\to\infty} \sum^{n}_{k=1}k(a_k-a_{k+1})
$$
Es gilt, dass $$
	\sum_{k=1}^n k(a_k-a_{k+1}) = \sum^n_{k=1}a_k - n\cdot a_{n+1}
$$
Beweis durch Induktion:\\
\textbf{Induktionsanfang}\\
$$	\sum_{k=1}^1 k(a_k-a_{k+1}) =a_1 - a_{2}$$

\textbf{Induktionsschritt}\\
Induktionsbehauptung: $$
	\sum_{k=1}^n k(a_k-a_{k+1}) = \sum^n_{k=1}a_k - n\cdot a_{n+1}
$$
\begin{align}
	\sum_{k=1}^{n+1} k(a_k-a_{k+1}) &  &                                                 \\
	=\sum_{k=1}^n k(a_k-a_{k+1})+(n+1)(a_{n+1}-a_{n+2})                                  \\
	\overset{IB}{=} \sum^n_{k=1}a_k - n\cdot a_{n+1}+n\cdot a_{n+1}+a_{n+1}-(n+1)a_{n+2} \\
	=\sum^{n+1}_{k=1}a_k - (n+1)\cdot a_{n+2}
\end{align}



Sei ein $\varepsilon >0$.\\
Da $\sum_n a_n$ konvergiert, gilt für diese Reihe das Cauchy-Kriterium.\\
Es kann also ein N gefunden werden, sodass für alle $m>n>N$ gilt
$$
	\sum_{k=n}^{m}a_k<\varepsilon
$$
Da $(a_n)_n$ eine monotone Nullfolge ist, sind entweder alle Folgenglieder negativ, oder alle Folgenglieder positiv.
Sei also oBdA $a_n\geq0, \forall n\in\mathbb{N} $.
\begin{align}
	\sum^n_{k=1}k(a_k-a_{k+1})-\sum^m_{k=1}k(a_k-a_{k+1})                   \\
	=\sum^n_{k=1}a_k-n\cdot a_{n+1}-(\sum^m_{k=1}a_k-m\cdot a_{m+1}       ) \\
	= \sum^m_{k=n}a_k - n\cdot a_{n+1} +m\cdot a_{n+1}                      \\
	= \sum^m_{k=n}a_k -( n\cdot a_{n+1} -m\cdot a_{n+1})                    \\
	< \varepsilon -(n\cdot a_{n+1}-m\cdot a_{n+1})<\varepsilon
\end{align}
Somit gilt das Cauchy-Kriterium für die Folge $\sum^n_{k=1}k(a_{k}-a_{k+1})$.\\

\textbf{Zusatz}\\
Da die Reihe eine Monoton steigende Folge ist,  gilt außerdem:
$$
	0<\sum_{k=n}^{m}k(a_k-a_{k+1})<\varepsilon -(n\cdot a_{n+1}-m\cdot a_{n+1})
$$
Somit gilt:
$$
	\varepsilon >(n\cdot a_{n+1}-m\cdot a_{n+1})=|(m\cdot a_{m+1}-n\cdot a_{n+1})|
$$
Die Folge $(n\cdot a_{n+1})_n$ erfüllt also ebenfalls das Cauchy-Kriterium und konvergiert.
\\
Nach dem Cauchy Verdichtungskriterium konvergiert die Reihe $\sum^n 2^n\cdot a_{2^n}$.

\\Die Folge $(2^n\cdot a_{2^n})_n$ ist also eine Nullfolge.\\
$$
	0<(2^{n+1}-1)a_{2^{n+1}}<2^{n+1}a_{2^{n+1}}
$$
Nach dem Sandwich-Satz konvergiert die Folge $((2^{n+1}-1)a_{2^{n+1}})_n$ gegen 0.\\
Da diese Folge eine Teilfolge der konvergenten Folge $(n\cdot a_{n+1})_n$ ist, muss die gesamte Folge gegen 0 konvergieren, da nach Bemerkung 4.16 alle Teilfolgen einer konvergenten Folge gegen den Grenzwert der Folge konvergieren.\\

Es gilt: \\
\begin{align}
	\sum^{\infty}_{n=1}n(a_n-a_{n+1}) =\lim \limits_{n\to\infty} \sum^{n}_{k=1}k(a_k-a_{k+1}) \\
	= \lim \limits_{n\to\infty} \sum^{n}_{k=1}a_k-\lim \limits_{n\to\infty}n\cdot a_{n+1}
\end{align}
Nach den Rechenregeln für konvergente Folgen gilt also:
$$
	\sum^{\infty}_{n=1}n(a_n-a_{n+1}) = \lim \limits_{n\to\infty} \sum^{n}_{k=1}a_k+0 =\sum^{\infty}_{n=1}a_n
	\qed
$$
\end{document}

